\documentclass{report}
\usepackage[utf8]{inputenc}
\usepackage{amsmath}
\usepackage{amsfonts}
\usepackage{listings}
\usepackage{graphicx}
\usepackage[table]{xcolor}
\usepackage{caption}
\usepackage{url}
\usepackage{fancyhdr}


\pagestyle{fancy}
\fancyhead{}
\fancyhead[L]{Johan Wikström - Anaël Bonneton}
\fancyfoot{}

\begin{document}
\title{Modeling the zombie apocalypse using a HIZD-model and HPC}
\author{Johan Wikström - 645714 \\
        Anaël Bonneton - 646275}
\maketitle
\tableofcontents

\section{Problem description}	

\paragraph{}
SIR-models (Susceptible, Infected, Removed/Recovered) are commonly used to model diseases in populations. In this report, we use a modified SIR-model called a HIZD-model (Human,Infected, Zombie, Decayed) to model a potential zombie apocalypse. The aim of the report is to determine the effect of some vital parameters, such as the infection risk when encountering a zombie, the incubation time for an infected human, on the survival of the human and zombie populations. In order to speed up the computation, we have used multi-threading and MPI\cite{openmpi}. We have run the simulation on an IBM iDataplex x86 system with up to 256 MPI processes.

\section{Model}

\paragraph{}
The simulation takes place in the Northern Territory in Australia. In order to model this geographical area we use a 1500*1000 mesh where each square represents 1 km\textsuperscript{2}. Each square can be inhabited or empty. There are three types of inhabitants : \emph{humans}, \emph{infected} and \emph{zombies}. This is a sparsely populated area but it is the case in the Northern Territory where the density is only 0.17/km\textsuperscript{2}. In our model, the time granularity is days. 

In our model we have a number of variable factors, some relatively known and some unknown that can be changed to model different outcomes of the zombie apocalypse.

\begin{itemize}
\item \emph{Human death rate}: The average human death rate per person, per day
\item \emph{Human birth rate}: The average birth rate per human per day
\item \emph{Brain eating probability}: The probability that a human is infected by a zombie of the collide, i.e. occupy the same square.
\item \emph{Infected to zombie probability}: The probability that an infected human will turn into a zombie on any given day.
\item \emph{Human move probability}: The probability that a human will move outside of a 1x1 km\textsuperscript{2} radius on any given day.
\item \emph{Zombie move probability}: The probability that a zombie will move outside of a 1x1 km\textsuperscript{2} radius on any given day.
\end{itemize}

\subsection{Humans}
\paragraph{}
The initial population statistics of our humans were based on actual Australian population data (table~\ref{AustralianData}). The humans in our model move in random directions, much like real humans do in a crisis, and this assumption has also been made in earlier works\cite{munz}. Humans can only move at a maximum pace of one square per day and they can only move in the directions up, down, left and right. When a human moves into a square inhabited by a zombie there is a probability that the human gets bitten and turns into a zombie. The probability of infection when in the same area as a zombie is currently unknown and in the model, we experiment with different values for this probability. In case of a zombie apocalypse, the authors recommend determining this parameter through empirical studies, placing a human and a zombie in the same square kilometer and measuring the time to infection.
\begin{table}[!h]
    \begin{tabular}{|l|l|}
      \hline
        Density    & 0.17/km\textsuperscript{2} \\
      \hline
        Birth rate & 13.3 per year per 1000 persons \\
      \hline
        Death rate & 6.5 per year per 1000 persons \\
      \hline
\end{tabular}
\caption{Northern Territory demographic data}\label{AustralianData}
\end{table}

\paragraph{}
Most of the humans' daily moving are within a kilometer. In our simulation, the probability for a human to travel more than 1 kilometer in one day is 0.4. When a human encounters a zombie, it may become infected with the probability of 0.65.

\subsection{Zombies}
The life of the zombie is largely unknown. We do not know how fast they move, how long before they decompose or how long the zombie incubation time is. For this reason, several parameters in out model can be adjusted to fit different apocalyptic scenarios.

\subsection{Infected humans}
The incubation time of the zombie virus is also largely unknown so this parameter is also variable in our model. During the incubation period an infected cannot infect other humans. However, the power of the zombie virus during the incubation time is so powerful that infected humans cannot die like humans, or decomposed like zombies. Judging by our sources\cite{deadheads}, a person bitten by a zombie gets a radically lowered risk of dying, almost always surviving until the point of zombiefication. After that, point however, the risk of dying by headshot is radically increased.




\section{Results}
%% TODO graphs !!!!!!!!!!!!!

%%% compare the evolution of the humans with and without the zombies
%%%     compare the population growth


%%% over the years, evolution of nb of humans, nb of zombies and infected
%%% per year (average) : number of infected

\section{Discussion of results}

\begin{thebibliography}{99}
\bibitem{openmpi}
Open MPI: Goals, Concept, and Design of a Next Generation MPI Implementation. Edgar Gabriel, Graham E. Fagg, George Bosilca, Thara Angskun, Jack J. Dongarra, Jeffrey M. Squyres, Vishal Sahay, Prabhanjan Kambadur, Brian Barrett, Andrew Lumsdaine, Ralph H. Castain, David J. Daniel, Richard L. Graham, and Timothy S. Woodall. In Proceedings, 11th European PVM/MPI Users' Group Meeting, Budapest, Hungary, September 2004
\bibitem{munz}
Munz, Philip, et al. "When zombies attack!: mathematical modelling of an outbreak of zombie infection." Infectious Disease Modelling Research Progress 4 (2009): 133-150.
\bibitem{deadheads}
Deadheads, 2011
\end{thebibliography}
\end{document}
