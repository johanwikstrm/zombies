\documentclass{report}
\usepackage[utf8]{inputenc}
\usepackage{amsmath}
\usepackage{amsfonts}
\usepackage{listings}
\usepackage{graphicx}
\usepackage[table]{xcolor}
\usepackage{caption}
\usepackage{url}
\usepackage{fancyhdr}


\pagestyle{fancy}
\fancyhead{}
\fancyhead[L]{Johan Wikström - Anaël Bonneton}
\fancyfoot{}

\begin{document}
\title{Modeling the zombie apocalypse using a HIZD-model and HPC}
\author{Johan Wikström - 645714 \\
        Anaël Bonneton - 646275}
\maketitle
\tableofcontents

\section{Problem description}	

\paragraph{}
SIR-models (Susceptible, Infected, Removed/Recovered) are commonly used to model diseases in populations. In this report, we use a modified SIR-model called a HIZD-model (Human,Infected, Zombie, Decayed) to model a potential zombie apocalypse. The aim of the report is to determine the effect of some vital parameters, such as the infection risk when encountering a zombie, the incubation time for an infected human, on the survival of the human and zombie populations.

\section{Model}

\paragraph{}
The simulation takes place in the Northern Territory in Australia. In order to model this geographical area we use a 1500*1000 mesh where each square represents 1 km\textsuperscript{2}. Each square can be inhabited or empty. There are three types of inhabitants : \emph{humans}, \emph{infected} and \emph{zombies}. This is a sparsely populated area but it is the case in the Northern Territory where the density is only 0.17/km\textsuperscript{2}. In our model, the time granularity is days. 

\subsection{Humans}
\paragraph{}
To model the humans' behaviour, we have used the Northern Territory demographic data (table~\ref{AustralianData}).
\begin{table}[!h]
    \begin{tabular}{|l|l|}
      \hline
        Density    & 0.17/km\textsuperscript{2} \\
      \hline
        Birth rate & 13.3 per year per 1000 persons \\
      \hline
        Death rate & 6.5 per year per 1000 persons \\
      \hline
\end{tabular}
\caption{Northern Territory demographic data}\label{AustralianData}
\end{table}

\paragraph{}
Most of the humans' daily moving are within a kilometer. In our simulation, the probability for a human to travel more than 1 kilometer in one day is 0.4. When a human encounters a zombie, it may become infected with the probability of 0.65.

\subsection{Zombies}
Zombies move slower than men. The probability for a zombie to travel more than 1 kilometer in one day is only 0.2. Zombies life is quite short : they decompose after about one months.

\subsection{Infected humans}
Infected humans become zombies after an incubation time of about 5 days. During the incubation period an infected cannot infect other humans. The behaviour of the infected is half-human, half-zombie : sometimes they move as quickly as humans, sometimes they are as slow as zombies.



To speed up the calculations, the mesh is divided into several smaller meshes and run on a Blue Gene Q using mpi. Each smaller mesh also uses multithreading to speed up individual computations.







\section{Results}
%% TODO graphs !!!!!!!!!!!!!

%%% over the years, evolution of nb of humans, nb of zombies and infected
%%% per year (average) : number of infected

\section{Discussion of results}

\end{document}
