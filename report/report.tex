\documentclass{report}
\usepackage[utf8]{inputenc}
\usepackage{amsmath}
\usepackage{amsfonts}
\usepackage{listings}
\usepackage{graphicx}
\usepackage[table]{xcolor}
\usepackage{caption}
\usepackage{url}
\usepackage{fancyhdr}


\pagestyle{fancy}
\fancyhead{}
\fancyhead[L]{Johan Wikström - Anaël Bonneton}
\fancyfoot{}

\begin{document}
\title{Modeling the zombie apocalypse using a HIZD-model and HPC}
\author{Johan Wikström - 645714 \\
        Anaël Bonneton - 646275}
\maketitle
\tableofcontents

\section{Problem description}	
SIR-models (susceptible, infected, removed/recovered) are commonly used to model diseases in populations. In this report, we use a modified SIR-model called a HIZD-model (human,infected,zombie,decayed) to model a potential zombie apocalypse. The aim of the report is determining the effect of some vital parameters, such as the infection risk when encountering a zombie, on the survival of the human and zombie populations.

\section{Model}
As a model of a geographical area we use a mesh of 1*1 km squares. Each square can be inhabited or empty and there are three types of inhabitants: humans, infected and zombies. This is a sparsely populated area but in the case of a zombie apocalypse, the authors believe that many areas will become sparsely populated rather quickly.

The time granularity is days and the modeled inhabitants may choose to move one square in a random direction or stay at the square they are inhabiting.

To speed up the calculations, the mesh is divided into several smaller meshes and run on a Blue Gene Q using mpi. Each smaller mesh also uses multithreading to speed up individual computations.


\paragraph{}
In our model we have considered three kinds of agents : \emph{humans}, \emph{infected humans}, and \emph{zombies}. These agents are described in the following subsections.

\subsection{Humans}
\paragraph{}
As the simulation takes place in the Northern Territory, in Australia, we have used the demographic data of this area (figure~\ref{AustralianData}).
\begin{figure}[!h]
    \begin{tabular}{|l|l|}
      \hline
        Density    & 0.17/km\textsuperscript{2} \\
      \hline
        Birth rate & 13.3 per year per 1000 persons \\
      \hline
        Death rate & 6.5 per year per 1000 persons \\
      \hline
\end{tabular}
\caption{Northern Territory demographic data}\label{AustralianData}
\end{figure}

\paragraph{}
Most of the humans' daily moving are within a kilometer. In our simulation, the probability for a human to travel more than 1 kilometer in one day is 0.4.

When a human encounters a zombie, or an infected human, it may become infected with the probability ....... %%TODO

\subsection{Infected humans}
An infected human becomes a zombie after an incubation period of about ............. days. %% TODO

\subsection{Zombies}
Zombies move slower than men. The probability for a zombie to travel more than 1 kilometer in one day is 0.2.
Zombies life is quite short : they decompose after about one months.




\section{Results}
%% TODO graphs !!!!!!!!!!!!!

%%% over the years, evolution of nb of humans, nb of zombies and infected
%%% per year (average) : number of infected

\section{Discussion of results}

\end{document}
