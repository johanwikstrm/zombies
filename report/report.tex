\documentclass{report}
\usepackage[utf8]{inputenc}
\usepackage{amsmath}
\usepackage{amsfonts}
\usepackage{listings}
\usepackage{graphicx}
\usepackage[table]{xcolor}
\usepackage{caption}
\usepackage{url}
\usepackage{fancyhdr}

\title{Modeling the zombie apocalypse using a HIZD-model and HPC}
\author{Johan Wikström - 645714}

\pagestyle{fancy}
\fancyhead{}
\fancyhead[L]{Johan Wikström - 645714}
\fancyfoot{}

\begin{document}
\maketitle
\tableofcontents
\section{Problem description}	
SIR-models (susceptible, infected, removed/recovered) are commonly used to model diseases in populations. In this report, we use a modified SIR-model called a HIZD-model (human,infected,zombie,decayed) to model a potential zombie apocalypse. The aim of the report is determining the effect of some vital parameters, such as the infection risk when encountering a zombie, on the survival of the human and zombie populations.
\section{Model}
As a model of a geographical area we use a mesh of 1*1 km squares. Each square can be inhabited or empty and there are three types of inhabitants: humans, infected and zombies. This is a sparsely populated area but in the case of a zombie apocalypse, the authors believe that many areas will become sparsely populated rather quickly.

The time granularity is days and the modeled inhabitants may choose to move one square in a random direction or stay at the square they are inhabiting.

To speed up the calculations, the mesh is divided into several smaller meshes and run on a Blue Gene Q using mpi. Each smaller mesh also uses multithreading to speed up individual computations.


\section{Results}

\section{Discussion of results}

\end{document}
